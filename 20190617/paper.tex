
\documentclass[12pt, a4paper]{article}
\usepackage[top=2cm, left=2cm, right=2cm, bottom=2cm]{geometry}
\usepackage{fontspec}
\setmainfont{Times New Roman}

\begin{document}
\title{\line(1,0){250} \\ \textbf{Distributed Representations of Words and Phrases and their Compositionality} \\  \line(1,0){250}}
\author{Tomas Mikolov}
\author{Jeffrey Dean}
\date{}
\maketitle


\begin{abstract}
The recently introduced continuous Skip-gram model is an efficient method for learning high-quality distributed vector representations that capture a large number of precise syntactic and semantic word relationships. In this paper we present several extensions that improve both the quality of the vectors and the training speed. By subsampling of the frequent words we obtain significant speedup and also learn more regular word representations. We also describe a simple alternative to the hierarchical softmax called negative sampling. 
\end{abstract}



\end{document}
