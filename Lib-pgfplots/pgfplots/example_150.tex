\documentclass{article}
% translate with >> pdflatex -shell-escape <file>

% This file is an extract of the PGFPLOTS manual, copyright by Christian Feuersaenger.
% 
% Feel free to use it as long as you cite the pgfplots manual properly.
%
% See
%   http://pgfplots.sourceforge.net/pgfplots.pdf
% for the complete manual.
%
% Any required input files (for <plot table> or <plot file> or the table package) can be downloaded
% at
% http://www.ctan.org/tex-archive/graphics/pgf/contrib/pgfplots/doc/latex/
% and
% http://www.ctan.org/tex-archive/graphics/pgf/contrib/pgfplots/doc/latex/plotdata/

\usepackage{pgfplots}
\pgfplotsset{compat=newest}

\pagestyle{empty}

\begin{document}
\begin{tikzpicture}
	\begin{axis}[
		3d box,
		zmax=1.4,
		colorbar,
		xlabel=$x$,
		ylabel=$y$,
		zlabel={$f(x,y) = x\cdot y$},
		title={Using Coordinate Filters to fix $z=1.4$}]
	% `pgfplotsexample4.dat' contains similar data as in 
	% `pgfplotsexample4_grid.dat', but it uses a uniform
	% matrix structure (same number of points in every scanline).
	% See examples above for extracts.
	\addplot3[surf,mesh/ordering=y varies] 
		table {plotdata/pgfplotsexample4.dat};
	\addplot3[scatter,scatter src=\thisrow{f(x)},only marks, z filter/.code={\def\pgfmathresult{1.4}}] 
		table {plotdata/pgfplotsexample4_grid.dat};
	\end{axis}
\end{tikzpicture}
\end{document}
