\documentclass{article}
% translate with >> pdflatex -shell-escape <file>

% This file is an extract of the PGFPLOTS manual, copyright by Christian Feuersaenger.
% 
% Feel free to use it as long as you cite the pgfplots manual properly.
%
% See
%   http://pgfplots.sourceforge.net/pgfplots.pdf
% for the complete manual.
%
% Any required input files (for <plot table> or <plot file> or the table package) can be downloaded
% at
% http://www.ctan.org/tex-archive/graphics/pgf/contrib/pgfplots/doc/latex/
% and
% http://www.ctan.org/tex-archive/graphics/pgf/contrib/pgfplots/doc/latex/plotdata/

\usepackage{pgfplots}
\pgfplotsset{compat=newest}

\pagestyle{empty}
\usepackage[pdftex]{pgfplots_ocg_copy}

\begin{document}
% requires \usepackage[pdftex]{ocg}
\begin{tikzpicture}
\begin{axis}[
	title=Dynamic PDF Layer Support (see Acrobat Layers),
	view={110}{35}]
\addplot3+[
	execute at begin plot visualization=\begin{ocg}{First Layer}{FirstLayer}{0},
	execute at end plot visualization=\end{ocg},
]
	coordinates {(0,0,12) (0,1,2) (1,0,6) (0,0,12)};

\addplot3+[
	execute at begin plot visualization=\begin{ocg}{Second Layer}{SecondLayer}{0},
	execute at end plot visualization=\end{ocg},
]
	coordinates {(0,0,9) (0,1,8) (1,0,4) (0,0,9)};

\addplot3+[
	execute at begin plot visualization=\begin{ocg}{Third Layer}{ThirdLayer}{0},
	execute at end plot visualization=\end{ocg},
]
	coordinates {(0,0,1) (0,1,7) (1,0,3) (0,0,1)};
\end{axis}
\end{tikzpicture}
\end{document}
