%!TEX root = thesis.tex

\prefacesection{Abstract}

Teaching machines to understand human language documents is one of the most elusive and long-standing challenges in Artificial Intelligence. This thesis tackles the problem of reading comprehension: how to build computer systems to read a passage of text and answer  comprehension questions. On the one hand, we think that reading comprehension is an important task for evaluating how well computer systems understand human language. On the other hand, if we can build high-performing reading comprehension systems, they would be a crucial technology for applications such as question answering and dialogue systems.

In this thesis, we focus on neural reading comprehension: a class of reading comprehension models built on top of deep neural networks. Compared to traditional sparse, hand-designed feature-based models, these end-to-end neural models have proven to be more effective in learning rich linguistic phenomena and improved performance on all the modern reading comprehension benchmarks by a large margin.

This thesis consists of two parts. In the first part, we aim to cover the essence of neural reading comprehension and present our efforts at building effective neural reading comprehension models, and more importantly, understanding what neural reading comprehension models have actually learned, and what depth of language understanding is needed to solve current tasks. We also summarize recent advances and discuss future directions and open questions in this field.

In the second part of this thesis, we investigate how we can build practical applications based on the recent success of neural reading comprehension. In particular, we pioneered two new research directions: 1) how we can combine information retrieval techniques with neural reading comprehension to tackle large-scale open-domain question answering; and 2) how we can build conversational question answering systems from current single-turn, span-based reading comprehension models. We implemented these ideas in the \sys{DrQA} and \sys{CoQA} projects and we demonstrate the effectiveness of these approaches. We believe that they hold great promise for future language technologies.
