% !TEX TS-program = pdflatexmk
\documentclass[12pt]{report}
\usepackage{suthesis}
%\documentstyle[12pt,suthesis]{report}

% -- Imports --
% (general libraries)
\usepackage{times,latexsym,amsfonts,amssymb,amsmath,graphicx,url,bbm,rotating}
\usepackage{multirow,hhline,stmaryrd,bussproofs,mathtools,siunitx}
\usepackage{booktabs,xcolor,csquotes}
% (custom libraries)
\usepackage{afterpage}
\usepackage{longtable}
\usepackage{fitch}
% (inline references)
\usepackage{natbib}
\usepackage{tabularx}
\usepackage[hidelinks]{hyperref}
\hypersetup{
    colorlinks=true,
    citecolor=.,
    linkcolor=.,
    urlcolor=blue
}

\usepackage{epigraph}
\renewcommand{\epigraphsize}{\normalsize}
\setlength{\epigraphwidth}{0.9\textwidth}

% (tikz)
\usepackage{soul}
\definecolor{light-yellow}{RGB}{255, 255, 153}
\sethlcolor{light-yellow}
\usepackage{tikz}
\usepackage{tikz-dependency,pifont}
\usetikzlibrary{shapes.arrows,chains,positioning,automata,trees,calc}
\usetikzlibrary{patterns,matrix}
\usetikzlibrary{decorations.pathmorphing,decorations.markings}
% (print algorithms)
\usepackage[ruled,lined,linesnumbered]{algorithm2e}
% (custom)
\input std-macros.tex
\input macros.tex

% (paper compilation hacks)
\def\newcite#1{\citet{#1}}
\def\cite#1{\citep{#1}}
%\def\newcite#1{\textcite{#1}}
%\def\cite#1{\autocite{#1}}
\definecolor{darkblue}{rgb}{0.0,0.0,0.4}


% Common hyphenations
\hyphenation{Text-Runner}
\hyphenation{Verb-Ocean}
\hyphenation{Raj-pur-kar}

%\bibliographystyle{plainnat}


% Comments
\usepackage{xspace}
\usepackage{xargs} % commandx
\usepackage[colorinlistoftodos,prependcaption,textsize=tiny]{todonotes}
\usepackage{marginnote}
\usepackage{color}
\definecolor{darkgreen}{RGB}{0,100,0}

% Inline comments useful for tables and figures.
\newcommandx{\icmtl}[2][1=]{\todo[inline]{DC: #2}\xspace}
\newcommandx{\icmtm}[2][1=]{\todo[inline]{CM: #2}\xspace}

% Comments for other places.
\newcommandx{\cmtl}[2][1=]{\todo[linecolor=blue,backgroundcolor=blue!10,bordercolor=blue,#1]{DC: #2}\xspace}
\newcommandx{\cmtm}[2][1=]{\todo[linecolor=red,backgroundcolor=red!10,bordercolor=red,#1]{CM: #2}\xspace}

\newcommand\cmb[1]{\marginpar{\tiny\raggedright\textcolor{blue}{\textsf{ DC\@: #1}}}}
\newcommand\cmm[1]{\marginpar{\tiny\raggedright\textcolor{red}{\textsf{\bfseries CM\@: #1}}}}

\usepackage{enumerate}

\setcounter{secnumdepth}{3}

\usepackage{footnote}
\makesavenoteenv{tabular}
\makesavenoteenv{table}

\usepackage{xpinyin}






\iffalse
% -- Document --
\begin{document}

% Title
\title{Neural Reading Comprehension and Beyond}
\author{Danqi Chen}
\principaladviser{Christopher D. Manning}
\firstreader{Dan Jurafsky}
\secondreader{Percy Liang}
\thirdreader{Luke Zettlemoyer}

% Preface
\beforepreface
\input preface.tex
\input ack.tex
\afterpreface
\hypersetup{linkcolor=magenta}


% -- Sections --
% Introduction
\chapter{Introduction}
\label{chapter:intro}
\input intro.tex

\part{Neural Reading Comprehension: Foundations}

\chapter{An Overview of Reading Comprehension}
\label{chapter:rc-overview}
\input chapters/rc_overview/intro.tex
\input chapters/rc_overview/history.tex
\input chapters/rc_overview/task.tex
\input chapters/rc_overview/discussions.tex

\chapter{Neural Reading Comprehension Models}
\label{chapter:rc-models}
\input chapters/rc_models/intro.tex
\input chapters/rc_models/feature_classifier.tex
\input chapters/rc_models/sar.tex
\input chapters/rc_models/experiments.tex
\input chapters/rc_models/advances.tex

\chapter{The Future of Reading Comprehension}
\label{chapter:rc-future}
\input chapters/rc_future/overview.tex
\input chapters/rc_future/datasets.tex
\input chapters/rc_future/models.tex
\input chapters/rc_future/questions.tex

\part{Neural Reading Comprehension: Applications}

\chapter{Open Domain Question Answering}
\label{chapter:openqa}
\input chapters/openqa/intro.tex
\input chapters/openqa/related_work.tex
\input chapters/openqa/system.tex
\input chapters/openqa/evaluation.tex
\input chapters/openqa/future.tex
% \input chapters/openqa/future.tex

\chapter{Conversational Question Answering}
\label{chapter:coqa}
\input chapters/coqa/intro.tex
\input chapters/coqa/related_work.tex
\input chapters/coqa/dataset.tex
\input chapters/coqa/models.tex
\input chapters/coqa/experiments.tex
\input chapters/coqa/discussions.tex

% Conclusion
\chapter{Conclusions}
\label{chapter:conclusions}
\input conclude.tex

% Bibliography
\bibliographystyle{acl_natbib_nourl}
\bibliography{ref}

\end{document}
\fi
