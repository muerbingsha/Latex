% Kalman filter system model
% by Burkart Lingner
% An example using TikZ/PGF 2.00
%
% Features: Decorations, Fit, Layers, Matrices, Styles
% Tags: Block diagrams, Diagrams
% Technical area: Electrical engineering


\documentclass[12pt, a4paper]{article}

\usepackage{graphics}
\usepackage{tikz}
\usetikzlibrary{decorations.pathmorphing} % fitting shapes to coordinates
\usetikzlibrary{fit}			% fitting shapes to coordinates
\usetikzlibrary{backgrounds} % drawing the background after the foreground


\usepackage[english]{babel}
\usepackage[T1]{fontenc}
\usepackage{lmodern}
\usepackage{amsmath}
\usepackage{amsthm}
\usepackage{amsfonts}



\begin{document}

\begin{figure}[h]
\centering

% 1 - define
% The state vector is represented by a blue circle.
% "minimum size" makes sure all circles have the same size
% independently of their contents.
\tikzstyle{state}=[circle, thick, minimum size = 1.2cm, draw = blue!80, fill = blue!25]
\tikzstyle{measurement}=[circle, thick, minimum size = 1.2cm, draw = orange!80, fill = orange!25]			
\tikzstyle{input}=[circle, thick, minimum size =1.2cm, draw = purple!80, fill = purple!25]
\tikzstyle{matrx}=[rectangle, thick, minimum size = 1cm, draw=gray!80, fill=gray!25]
\tikzstyle{noise}=[circle, thick, minimum size=1.2cm, draw=yellow!80!black, fill=yellow!50, decorate, decoration={random steps, segment length=2pt, amplitude=2pt}]
\tikzstyle{background}=[rectangle, fill=gray!10, inner sep=0.2cm, rounded corners=5mm]



% 2 - draw			
\begin{tikzpicture}[>=latex, text height = 1.5ex, text depth = 0.25ex]

	% 2 - 1 nodes
	\matrix[row sep = 0.5cm, column sep = 0.5cm]{
		% First line: Control input
		&
		\node[input] 	(u_k-1) 	{$\mathbf{u}_{k-1}$};		&
		& 
		\node[input]	 (u_k)	{$\mathbf{u}_{k}$}; 		& 
		& 
		\node[input] 	(u_k+1)	{$\mathbf{u}_{k+1}$}; 	&
		\\
		
		% Second line
		\node[noise] 	(w_k-1)	 {$\mathbf{w}_{k-1}$}; 	&
		\node[matrx]	 (B_k-1)	{$\mathbf{B}$};			&
		\node[noise]	 (w_k)	{$\mathbf{w}_{k}$}; 		&
		\node[matrx]	(B_k)	{$\mathbf{B}$};			&
		\node[noise]	 (w_k+1)	{$\mathbf{w}_{k+1}$}; 	&
		\node[matrx] 	(B_k+1)	{$\mathbf{B}$};			&
		\\
		
		% Third line: State & state transition matrix
		\node		(A_k-2)	{$\cdots$};			&
		\node[state] 	(x_k-1)	{$\mathbf{x}_{k-1}$}; 	&
		\node[matrx]	(A_k-1)	{$\mathbf{A}$}; 			&
		\node[state] 	(x_k)		{$\mathbf{x}_{k}$}; 		&
		\node[matrx] 	(A_k)	{$\mathbf{A}$};			&
		\node[state]	(x_k+1)	{$\mathbf{x}_{k+1}$}; 	&
		\node		(A_k+2)	{$\cdots$};			&
		\\
		
		% Forth line
		\node[noise]	(v_k-1)	{$\mathbf{v}_{k-1}$}; 	&
		\node[matrx]	(H_k-1)	{$\mathbf{H}$}; 		&
		\node[noise]	(v_k)		{$\mathbf{v}_{k}$}; 		&
		\node[matrx]	(H_k)	{$\mathbf{H}$}; 		&
		\node[noise]	(v_k+1)	{$\mathbf{v}_{k+1}$}; 	&
		\node[matrx]	(H_k+1)	{$\mathbf{H}$}; 		&
		\\

		% Fifth line: Measurement
		& 
		\node[measurement](z_k-1){$\mathbf{z}_{k-1}$}; 	& 
		&
		\node[measurement](z_k){$\mathbf{z}_{k}$}; 		&
		& 
		\node[measurement](z_k+1){$\mathbf{z}_{k+1}$}; 	&
		\\
	};
	
	
	% 2 - 2 arrows
	\path[->]
		(A_k-2) 	edge[thick] 	(x_k-1)
		(x_k-1)	edge[thick]	(A_k-1)
		(A_k-1)	edge[thick]	(x_k)
		(x_k)		edge[thick]	(A_k)
		(A_k)	edge[thick]	(x_k+1)
		(x_k+1)	edge[thick]	(A_k+2)
		
		(w_k-1)	edge			(x_k-1)
		(w_k)	edge			(x_k)
		(w_k+1)	edge			(x_k+1)
		
		(v_k-1)	edge			(z_k-1)
		(v_k)		edge			(z_k)
		(v_k+1)	edge			(z_k+1)
		
		(u_k-1)	edge[thick]	(B_k-1)
		(u_k)		edge[thick]	(B_k)
		(u_k+1)	edge[thick]	(B_k+1)
		(B_k-1)	edge[thick]	(x_k-1)
		(B_k)	edge[thick]	(x_k)
		(B_k+1)	edge[thick]	(x_k+1)
		(x_k-1)	edge[thick]	(H_k-1)
		(x_k)		edge[thick]	(H_k)
		(x_k+1)	edge[thick]	(H_k+1)
		(H_k-1)	edge[thick]	(z_k-1)
		(H_k)	edge[thick]	(z_k)
		(H_k+1)	edge[thick]	(z_k+1)
		;
		
	% 2 - 3 background
	\begin{pgfonlayer}{background}
		\node [background, fit = (u_k-1) (u_k+1), label = left: Entrance:] {};
		\node [background, fit = (w_k-1) (H_k+1)] {};
		\node [background, fit = (z_k-1) (z_k+1), label = left: Measure:] {};
		
	\end{pgfonlayer}
	
\end{tikzpicture}



\end{figure}


\end{document}