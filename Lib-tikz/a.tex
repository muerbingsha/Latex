\documentclass[a4paper]{article}

\usepackage{tikz}
\usetikzlibrary{datavisualization}

\begin{document}

\begin{tikzpicture}
	\datavisualization data group {lines} = {
 		data point [x=0, y=0, set=normal]
		data point [x=2, y=2, set=normal]
		data point [x=0, y=1, set=heated]
		data point [x=2, y=1, set=heated]
		data point [x=0.5, y=1.5, set=critical]
		data point [x=2.25, y=1.75, set=critical]
		};
  	\datavisualization [ school book axes={unit=0.3},
		visualize as line=normal,
		visualize as line=heated,
		visualize as line=critical,
		normal={style={green}, label in legend={text={normal}}},
		heated={style={yellow}, label in legend={text={heated}}},
		critical={style={red}, label in legend={text={critical}}},
		legend=north east inside]
  		data group {lines};
  
 	\datavisualization [school book axes, visualize as smooth line]
		data [format=function]{
			var x: interval[-1.5:1.5] samples 7;
			func y = \value x*\value x;
		};
 
\coordinate (O) at (0, 0);
\coordinate (P1) at (1, 0);
\coordinate (P2) at (60:1cm);
\coordinate (P3) at (2*60:1cm);
\coordinate (P4) at (3*60:1cm);
\coordinate (P5) at (4*60:1cm);
\coordinate (P6) at (5*60:1cm);

\draw (P1)--(P2)--(P3)--(P4)--(P5)--(P6)--cycle;
\draw [color=gray] (O)--(P1) (O)--(P2) (O)--(P3) (O)--(P4) (O)--(P5) (O)--(P6);

\draw (O) node [below]{O};
\draw (P1) node [right]{P1};
\draw (P2) node [above right]{P2};
\draw (P3) node [above left]{P3};
\draw (P4) node [left]{P4};
\draw (P5) node [below left]{P5};
\draw (P6) node [below right]{P6};



	

\end{tikzpicture}


\begin{tikzpicture} [> = stealth, % arr

\end{tikzpicture}

\end{document}

